% (This is included by thesis.tex; you do not latex it by itself.)

\begin{abstract}

This paper provides several contributions related to the workpiece drainability problem in manufacturing. First we describe our parametric approach to a more accurate kinetic model that models particles as parabolic path segments rather than rays. This improvement allows for particles to maintain velocity, experience changing acceleration fields, and collide with workpiece surfaces while maintaining constant-time geometric primitive intersecton tests.

Next, we provide a framework for approaching the drainability problem from an artificial intelligence perspective. We define our state space formulation and our sampling approach to the successor function which produces information on the connectivity between vertices of the workpiece.

Finally, we describe how to obtain both a solution to a given draining problem and the full control sequence of the workpiece rotator which is immediately applicable in industry.

%First we describe our method for
%This improvement leads to a technique for handling vertex placement
%Next, we rpovide
%Finally, we describe our method
%We survey the computational geometry relevant to, we especially focus on, we briefly survey
%We describe data structures for representing... our implementation of the data structure...

\vspace{0.2in}

\textbf{Keywords}: drainability, parametric equations, fluid simulation

\end{abstract}
