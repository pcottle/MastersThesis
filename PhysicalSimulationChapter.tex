						\chapter{Physical Simulation of Water Particles}\label{physicalChapter}

Water particle simulation is the underlying core of drainability analysis -- the behavior of the fluid (or water particles in this case) must be simulated in order to analyze when and where the fluid can leave the workpiece.

Here we present the approach to simulating straight-line water particle motion based on parametric rays (the approach used in Yasui et al.'s work) and then our modification of this approach to fit parabolic water particles.

\section{Reduced Ray Approach}\label{KineticEquation}

	\subsection{Underlying Kinetic Model}

Individual water particles under an acceleration field $a$ with initial position $x_0$ and initial velocity $v_0$ can be modeled with a simple kinetic equation.

\myequation{
	x(t) = x_0 + v_0 \cdot t + \frac{1}{2} a \cdot t^2
}{
	\label{eq:basickinematic}
}

Equation \eqref{eq:basickinematic} shows the basic kinematic model of a water particle.


	\subsection{Reduction to Rays}

In order to obtain particles that always travel in straight lines, one may omit the acceleration term from Equation \eqref{eq:basickinematic}. This is the approach that Yasui et al. chose to obtain straight-line particle motion \cite{Yasui2011}. Under this condition, these particles can effectively be modeled as ``rays.''

\myequation{
x(t) = x_0 + v_0 \cdot t
}{
	\label{eq:kinematicDropAccel}
}

Equation \eqref{eq:kinematicDropAccel} shows this simplified ray model. Once the motion of a particle under un-obstructed movement can be easily produced, the primary challenge of particle simulation is finding the collision points of a particle's path.

	\subsection{Parametric Equations (rays)}

There are many ways to find these intersection (or ``collision'') points. One common approach is to model the ray equation as a parametric equation \cite{glassner1989introduction} \cite{fundamentals}.

\myequation{
	\vec{x}(t) = \vec{x}_0 + \vec{v}_0 \cdot t
} {
	\label{eq:parametricRay}
}

This modeling choice is shown in Equation \eqref{eq:parametricRay} where $\vec{x}(t)$ describes the location of the particle at time $t$.

\myfigure{width=0.9\linewidth}{rayexample}{A visual depiction of a ray's component vectors and its path through time}

	\subsection{Ray Tracing}

Once particles are modeled as parametric rays, all the existing techniques and libraries from ``ray-tracing'' (a standard approach to producing 3D computer graphics) can be used to find intersection points \cite{glassner1989introduction} \cite{fundamentals}.

Ray-tracing produces 3D computer graphics by sending out rays from a camera location. If one of these rays intersects a geometric primitive in the scene, the resulting color of that ray is calculated and stored in a pixel table. These ray-primitive intersections are the fundamental computational bottleneck for ray tracing; consequently, much work has been done to optimize intersection tests. Parametric modeling of geometric primitives combined with parametric rays is the industry standard \cite{glassner1989introduction} \cite{fundamentals}.

%	\subsection{Geometric Primitive Intersections}

%Because these rays are defined parametrically, they can be substituted into the parametric definition of a geometric primitive to obtain exact solutions of intersection points. This is the primary advantage of parametric equations over integration schemes -- intersections between a ray and a geometric primitive can be solved for directly in constant time \cite{g||lassner1989introduction}.

%For example, the parametric definition of a sphere with center $C$ and radius $r$ is given in Equation \eqref{eq:sphereParametric}.

% \myequation{
% 	|\vec{X} - \vec{C}|^2 - r^2 = 0
% } {
% 	\label{eq:sphereParametric}
% }

% In Equation \eqref{eq:sphereParametric}, all points $\vec{X}$ that satisfy the equation define the surface of the sphere. In order to solve for the intersection of a ray and a sphere, the parametric equation of the ray is substituted into Equation \eqref{eq:sphereParametric} for $X$. This produces Equation \eqref{eq:sphereParametricSubbed}.

% \myequation{
% 	|(\vec{x}_0 + \vec{v}_0 \cdot t) - \vec{C}|^2 - r^2 = 0
% } {
% 	\label{eq:sphereParametricSubbed}
%}

% \myfigure{width=0.5\linewidth}{raysphere}{A demonstration of a ray intersecting with a sphere.}

% Equation \eqref{eq:sphereParametricSubbed} can be expanded algebraically to yield an equation that is quadratic in $t$. This resulting equation can then be solved using the standard quadratic formula:

% \myequation{
% 	\frac{-b \pm \sqrt{b^2 - 4ac}}{2a} = t
% } {
% 	\label{eq:quadratic}
% }

% This constant-time solution for the intersection between a parametric equation and geometric primitive will be crucial for later parts of this paper.

\section{Kinetic Parametric Approach}

While this straight-line particle approach is convenient, our model aims to examine particles with more realistic kinetic properties and paths. Consequently, we choose to not omit the acceleration term from the kinetic model of a particle. This means we combine a parametric approach with Equation \eqref{eq:basickinematic} to produce Equation \eqref{eq:freeFall} -- a parabola defined parametrically.

\myequation{
	\vec{x}(t) = \vec{x}_0 + \vec{v}_0 \cdot t + \frac{1}{2} \vec{a} \cdot t^2
}{
	\label{eq:freeFall}
}


	\subsection{Geometric Primitive Intersections}

We would like to intersect these particle paths (represented by parabolas) with geometric primitives in the workpiece for simulation purposes. Since our algorithm takes in a set of polygons that define the workpiece, the only geometric primitive of interest is a polygon edge. We will first describe the parabola-line intersection test and then present the small amount of extra validation needed for the parabola-edge intersection test.

We begin by redefining the parabola's initial point $\vec{x}_0$ as $\vec{p}$ and initial velocity $\vec{v}_0$ as $\vec{v}$ for notation purposes:

$$
\vec{x}(t) = \vec{p} + \vec{v} \cdot t + \frac{1}{2} \vec{a} \cdot t^2.
$$

This gives us a parabola defined by three vectors:

\myfigure{width=0.85\linewidth}{parablinetest2}{The parabola redefined for notation purposes.}

This parabola needs to be intersected against an arbitrary edge with starting point $\vec{s}$, direction vector $\vec{d}$, and length $|\vec{d}|$:

\myfigure{width=0.6\linewidth}{parablinetest1}{Edge for intersection test.}

We know that any point $\alpha$ on the line defined by $\vec{d}$ and $\vec{s}$ has a cross product of 0:

\myfigure{width=0.6\linewidth}{parablinetest3}{Definition of a point on the line defined by $\vec{d}$ and $\vec{s}$.}

This can be represented algebraically as

$$
(\vec{\alpha} - \vec{s}) \times \vec{d} = 0.
$$

Knowing this, we can substitute our definition of the particle's path through time into this cross product:

$$
(\vec{x}(t) - \vec{s}) \times \vec{d} = 0.
$$

We can then be expanded algebraically into:

$$
(\vec{p} + \vec{v} \cdot t + \frac{1}{2} \vec{a} \cdot t^2 - \vec{s}) \times \vec{d} = 0.
$$

In order to obtain a non-vectorized equation, we then break each vector into orthogonal components as shown in Equation \eqref{eq:brokenUp}.

$$
\vec{s} = (s_x, s_y) \qquad \vec{d} = (d_x, d_y)
$$

\myequation{
	\vec{p} = (p_x, p_y) \qquad \vec{v} = (v_x, v_y) \qquad \vec{a} = (a_x, a_y)
}{
  \label{eq:brokenUp}
}

This gives us two versions of the above equation, where the cross product operation is replaced by simple multiplication:

$$
\left [ (p_x - s_x) + v_x  t + \frac{1}{2} a_x  t^2) \right ] \cdot  d_y = 0
$$

\vspace{0.1in}

$$
\left [ (p_y - s_y) + v_y t + \frac{1}{2} a_y t^2) \right ] \cdot d_x = 0.
$$

Subtracting the second equation from the first to perform the cross product and grouping terms by the power of $t$ gives us:


\myequation{
	(\frac{1}{2} a_x d_y - \frac{1}{2} a_y ) \cdot t^2 + (v_x d_y - v_y d_x) \cdot t +
\left [ (p_x - s_x) d_y - (p_y - s_y) d_x \right ] = 0
} {
	\label{eq:parabLineTest}
}

Equation \eqref{eq:parabLineTest} can easily solved by the quadratic equation:

$$
	\frac{-b \pm \sqrt{b^2 - 4ac}}{2a} = t.
$$

Equation \eqref{eq:abc} defines variables $a$, $b$, and $c$.

\myequation{
	a = (\frac{1}{2} a_x d_y - \frac{1}{2} a_y d_x) \qquad b = (v_x d_y - v_y d_x) \qquad c = \left [ (p_x - s_x) d_y - (p_y - s_y) d_x \right ]
}{
	\label{eq:abc}
}

Solving for $t$ will yield two values, representing points along the parabola path that intersect the line defined by $\vec{d}$ and $\vec{s}$:

\myfigure{width=0.7\linewidth}{parablinetest4}{Solution points for the parabola-line intersection test.}

This concludes the parabola-line intersection tests. For our simulation purposes, we would like to determine the first point that lies on the edge defined by $\vec{d}$ and $\vec{s}$ rather than the line. This requires a small extra validation step; first, negative values of $t$ are eliminated, for these points represent the particle traveling backwards in time. Then, the smallest value of $t$ is found that produces a point that lies in between points $\vec{s}$ and $\vec{s} + \vec{d}$. If no such value exists, this parabola does not intersect the given edge.

\section{Equation Types}

Now that we have defined a parabolic path segment in parametric form and demonstrated a geometric primitive intersection test, we would like to use this equation to simulate particle movement over time. During the course of simulation however, a particle may be traveling in a variety of conditions which require modifications to Equation \eqref{eq:freeFall}. The two primary conditions that determine the modifications necessary are:

\begin{itemize}
	\item whether the workpiece is rotating,
	\item whether the particle is traveling along a workpiece edge or through open space.
\end{itemize}

These two conditions, when combined in all possible ways, produce four scenarios of particle simulation that must be modeled. The following sections cover each scenario and detail the equation used.


\section{No Rotation}

We first examine the two scenarios where no rotation of the workpiece occurs during simulation.

		\subsection{Scenario \#1: Free Fall Equation}

When the particle is moving through open space with no workpiece rotation, we consider that particle to be in basic ``free-fall.'' The particle travels through empty space subject to a constant acceleration field. The resulting motion is a parabola, and no modification to Equation \eqref{eq:freeFall} is necessary.

\myfigure{width=0.9\linewidth}{freefallequation}{Example path traced out by the ``free-fall'' equation}

		\subsection{Scenario \#2: Sliding Equation}

During simulation, a particle may also begin to travel along an edge. This usually occurs after a particle collides with an edge multiple times in a row and obtains a velocity that is parallel to the edge. Our simulation algorithm performs this test by checking if the velocity component perpendicular to the edge becomes smaller than some pre-defined $\epsilon$ value; if this is true, the particle is now under a ``sliding'' scenario.

In this scenario, the acceleration vector is projected along the surface of the workpiece. If this projected acceleration is zero, the particle will stay in place; if not, the particle will begin to ``slide'' along this edge.

We achieve the modeling of this scenario by projecting the ``free-fall'' acceleration in Equation \eqref{eq:freeFall} along the workpiece edge, which produces a straight-line path.

Note that while the particle is now traveling along an edge, the resulting equation takes on the same form as the free-fall equation; the only difference is that the acceleration is now projected. Recall that $\vec{x}(t)$ here describes the position of the particle at time $t$.

\myequation{
	\vec{x}(t) = \vec{x}_0 + \vec{v}_{0} \cdot t + \frac{1}{2}\vec{a}_{projected} \cdot t^2
}{
	\label{eq:sliding}
}

\myfigure{width=0.9\linewidth}{slidingequation}{Demonstration of the sliding equation.}

With these two formulations, the particle's path can be simulated on both edges and in open space when the acceleration field is constant (e.g. the workpiece is not rotating). Furthermore, because both of these equations are strictly quadratic in $t$, they can be substituted into the definition of geometric primitives to perform intersect tests.

\section{Rotation}

Here we begin to examine the scenarios where the workpiece is rotating during particle simulation.

In this paper, we choose our frame of reference to be the X and Y axes that define the workpiece geometry. This means that when the workpiece rotates, our frame of reference stays fixed to the workpiece. Consequently, rotation changes the orientation of the acceleration field.

We see now how this rotating acceleration field affects the following scenarios. While we originally presented the edge travel scenario (Scenario \#2) second in the ``No Rotation'' section, we will present the edge travel scenario first in the ``Rotation'' section. This will illuminate the challenges we face when simulating the last scenario presented.

		\subsection{Scenario \#3: Concurrent Rotation \& Sliding Equation}

In this scenario, the particle is traveling along a workpiece edge and the workpiece is rotating.

We know that the acceleration vector is projected along the edge during edge travel. This means that the direction of the acceleration in this scenario is fixed; only the magnitude of the projected acceleration varies as the workpiece rotates. Consequently, rotation of the acceleration field only results in the magnitude of the projected acceleration changing.

This leads to a special case of particle motion which we will refer to as ``concurrent rotation and sliding.'' We first define $\hat{a}_{projected}$ as a constant unit-vector in the direction of the edge and $|a_{projected}|(t)$ as the magnitude of the projected acceleration over time. With these two new variables, we can show the resulting formulation of this scenario in Equation \eqref{eq:slidingRotating}.

\myequation{
	\vec{x}(t) = \vec{x}_0 + \vec{v}_{0} \cdot t + \frac{1}{2}\hat{a}_{projected}  \cdot t^2 \cdot |a_{projected}|(t)
} {
	\label{eq:slidingRotating}
}

This scenario is further illustrated in Figure \ref{scenario4}.

\myfigure{width=0.6\linewidth}{scenario4}{Illustration of Scenario \#4 -- concurrent rotation and free-fall.}

If rotation occurs at a constant rate and the acceleration vector begins perpendicular to the edge, the magnitude term can be modeled as

$$
mag_{accel} = cos(\omega \cdot t)
$$

where $\omega$ is the angular velocity.

%		\subsubsection{Concurrent Rotation \& Sliding Equation Intersection}

In Equation \eqref{eq:slidingRotating}, we no longer have a formulation of particle motion that is strictly quadratic in $t$. The trigonometric terms introduced by the rotation of the workpiece make it impossible to solve for path-primitive intersections in constant time. We can, however, easily substitute in values of $t$ to determine the position of the particle at a given time.

Despite the inability to intersect these trajectories with geometric primitives in constant time, we still simulate this scenario in practice. The reason for this will be detailed after Scenario \#4 is presented, when a holistic comparison of these two scenarios can be conducted.

		\subsection{Scenario \#4: Concurrent Rotation \& Free-Fall Equation}

The fourth scenario is when the particle is traveling through open space and the workpiece is rotating. In this case, the acceleration vector is no longer projected along an edge as shown in Equation \eqref{eq:slidingRotating}. The acceleration vector now has a direction that changes in time (with a fixed magnitude), as shown in Equation \eqref{eq:freefallRotating}.

\myequation{
	\vec{x}(t) = \vec{x}_0 + \vec{v}_{0} \cdot t + \frac{1}{2}\vec{a}(t)  \cdot t^2
} {
	\label{eq:freefallRotating}
}

Here, the $\vec{a}(t)$ term describes the acceleration vector over time as the workpiece rotates. Assuming a constant rotation, this term breaks down to:

$$
a_x = cos(\alpha t) \cdot |a_g|;
$$

$$
a_y = sin(\alpha t) \cdot |a_g|.
$$

This scenario produces paths that mathematically are not parabolic, as illustrated below in Figure \ref{norotationfreefall}.

\myfigure{width=0.9\linewidth}{norotationfreefall}{Demonstration of the particle path during concurrent rotating and free-fall}

		\subsubsection{Assumption \#1 - No concurrent Rotation and Free-fall}

These non-linear trigonometric terms produce an equation that can no longer be intersected against a geometric primitive in constant-time. Intersection points can still be calculated for these path segments via approaches like Newton's method, but doing so requires additional computational time. Our approach to the draining problem requires many different particle simulations; consequently, we restrict our simulation scenarios to having constant-time intersection tests.

This restriction effectively prevents our ability to simulate particles in this ``concurrent rotation and free-fall'' scenario, leading us to our induced first assumption and restriction in this paper -- particles may not travel through open space while the workpiece is rotating.

Our results chapter provides quantitative analysis of this scenario \#4 rejection and a related justification.

		\subsubsection{Resolution of Scenario \#3}

The third scenario, concurrent rotation and sliding, similarly has a kinetic path equation that cannot be easily intersected with geometric primitives. The key difference between this scenario and scenario \#4, however, is that the particle is traveling along a known edge. Since this edge length is known as well as the duration of the turn, this particle path does \emph{not} need to be intersected with other primitives in the workpiece if it remains on the edge.

Instead, the position of the particle after the specified turn is calculated. If the particle is still on the same edge at the end of the turn, we can then proceed with simulation because we are guaranteed that the particle did not collide with any other surface during the turn. If the particle is beyond the length of the edge at the end of the turn, we do not have this no-collision guarantee and cannot proceed with simulation.

Thus although we have two scenarios where particle path and solid intersection tests are difficult, only one of them (concurrent rotation and free-fall) cannot be simulated at all. The other (concurrent rotation and sliding) can be simulated whenever the particle has not traveled beyond the edge at the end of the turn, which was the most common result during our tests.

	\section{Collisions}

Collisions occur when the path for a given particle intersects a surface inside the workpiece.

\myfigure{width=0.4\linewidth}{planarcollision}{A particle colliding with a surface and the resulting elastic collision}

After a collision occurs, the particle ``transitions'' into one of three states:

\begin{itemize}
\item the particle has come to rest and simulation terminates,
\item the particle begins a new path through open space,
\item ehe particle begins a new path along the surface of a polygon.
\end{itemize}

There are many types of collisions depending on the direction of the acceleration vector, the kinetic energy that the particle contains, and the angle between surfaces on a polygon. We present the two primary collisions (or transitions) of concern:

		\subsubsection{Planar To Sliding Transition}

When a particle collides with an edge and obtains a resulting perpendicular velocity ($v_{perp}$) less than some epsilon ($\epsilon$) value, the particle transitions from free-fall to sliding along an edge.


\myfigure{width=0.5\linewidth}{planartoslidingtransition}{Particle path transition from free-fall to sliding.}


		\subsubsection{Sliding To Free-Fall Transition}

When a particle reaches the end of an edge (effectively intersecting the adjacent edge), it may enter free-fall, where it is further simulated.

\myfigure{width=0.5\linewidth}{slidingtofreefalltransition}{Particle path transition from sliding to free-fall.}

%		\subsubsection{Sliding to Planar Transition}

%		When sliding along an edge, the particle may encounter another edge. If this edge has a dot product greater than zero, it collides with the edge and enters freefall again.

%\myfigure{width=0.5\linewidth}{slidingedgecollision}{Sliding particle collision with adjacent edge at obtuse.}

%		\subsubsection{Sliding-Corner Collision}

%		If the next edge has a dot product less than or equal to 0, the particle is effectively ``trapped'' as long as the gravity vector points within the edge.

%\myfigure{width=0.5\linewidth}{slidingcornercollision}{Sliding particle collision with adjacent edge at acute angle.}

	\subsection{Conservation of Momentum}

		The only source of energy in this demo is the potential energy from gravity. Note that as the workpiece rotates, the gravity vector (and corresponding field) rotates. This means that a particle in a low-energy configuration can transition to a high-energy configuration through rotation; consequently, rotation may add energy into the system.

		\subsubsection{Settling Guarantee}

		The only addition of energy is from rotation, so in the absence of rotation, no energy is added to the system. Because the elasticity  $\kappa$ of the particle-edge collisions is chosen to be less than 1 in simulation, energy dissipates as a particle collides with other surfaces.

		This leads us to the settling guarantee: the simulation of a particle during a time period with no rotation is guaranteed to terminate in one of two ways:

\begin{itemize}
\item the particle settles into a concave vertex with a kinetic energy less than $\epsilon$, or
\item the particle exits the workpiece, which is the goal of the search.
\end{itemize}

