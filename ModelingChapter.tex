						\chapter{Modeling Approach}

Analyzing the drainability of a workpiece requires that the behavior of fluid throughout the workpiece be modeled in some way. This behavior is complex; trapped gases, viscuous effects, and fluid mechanics all interact during workpiece draining. There are many ways to model this fluid behavior, and each choice comes with both advantages and disadvantages.

Advanced fluid simulation methods like the finite element method or smoothed-particle hydrodynamics approximate fluid behavior with high accuracy, but their computation is quite expensive in both time and memory. This is primarily because they are integration methods -- their accuracy relies on integrating many times over a small timestep.

Thus, in order to quickly simulate fluid behavior during drainability analysis, most previous work chooses a greatly simplified fluid model. While accuracy is sacrificed by choosing a simpler model, more useful analysis can be produced by simulating many different scenarios.

\section{Straight-line Multiple-Particle Model}

TODO -- something here?

	\subsection{Core Assumptions}

Yasui et. all \cite{plot} chose a simplified particle modeling approach with a few core underlying simplifications:

\begin{itemize}
	\item Each fluid body within the workpiece is approximated as a single particle
	\item Water particles only travel in straight lines and do not maintain a kinetic state
	\item The workpiece would rotate infinitesimally slow in one direction for an infinite amount of time
\end{itemize}

	\subsection{Modeling Choice Advantages and Disadvantages}

This modeling approach allowed Yasui et. all to reduce the fluid draining problem down to a particle draining problem. Furthermore, particle simulation within the workpiece is greatly simplified; particles always travel in straight lines and do not accumulate kinetic energy. This means that all collisions are completely inelastic, and particles leaving an edge travel parallel to the acceleration field.

Thus, particle simulation in this choice of model was essentially reduced to the intersection between lines and surfaces. Because these tests are computationally inexpensive and have a sub-linear runtime in the number of triangles within the workpiece, Yusuke et. all were able to perform many particle simulations during analysis. The end result is that for every possible rotation axis, a particle is simulated out of every concave vertex in the workpiece.

This thoroughness of simulation produces a set of results that describes the workpiece drainability of any rotation axis. This analysis serves great utility during the design stage of product creation; engineers can quickly determine if a geometry change would prevent drainability and ``seal'' off the part.

Yusuke et. all's model assumes infinitesimally slow rotation for an infinite amount of time; their results produce a binary value of drainability about a particular axis. If these results could be supplemented with a direct control sequence of how to drain the part for a particular rotation axis, manufacturers could further optimize the draining stage of product manufacturing.

This work aims to produce those complementary results. Furthermore, this work expands the fluid behavior model

%	\subsection{Infinitesimally Slow Rotations}

%The assumption about workpiece rotation meant that when particles rolled out of a concave vertex, the gravity direction was always nearly perpendicular to the leading edge.

%\myfigure{prevworkgravityperp}{The gravity direction is perpendicular to concave vertex leading edges.}

%	\subsection{Inelastic Collisions}

%Since water particles do not maintain kinetic state in Yasui et all, collisions against a plane are essentially just projections onto that plane.

%\myfigure{prevworkinelasticcollision}{Demonstration of purely inelastic collisions, where velocities are projected onto the collision surface.}

\section{Kinetic Single-Particle Model}

Our work addresses these concerns by utilizing a different (but related) model of the workpiece draining problem. The results produced are complementary to Yusuke et all's results and can be used in conjunction.

The objectives of a more accurate kinetic model and bi-directional draining lead to the following assumptions:

\begin{itemize}
	\item The fluid in the entire workpiece will be approximated by a single particle starting at a concave vertex
	\item Water particles travel according to a reasonable kinetic model (acceleration due to gravity, but no aerodynamic drag)
	\item The workpiece can be rotated in either direction along it's rotation axis
	\item The workpiece is a two dimensional polygonal mesh
\end{itemize}

	\subsection{Assumption Effects}

This modeling choice requires a different approach to drainability analysis than the approach used by Yusuke et all; our approach, however, draws inspiration from their previous work.

The key componenets of our approach break down into the water particle simulation (presented in Chapter 4) and the search for a drainability sequence (presented in Chapter 5).



