						\chapter{Modeling Approach}

Analyzing the drainability of a workpiece requires that the behavior of fluid throughout the workpiece be modeled in some way. This behavior is complex; trapped gases, viscuous effects, and fluid mechanics all interact during workpiece draining. There are many ways to model this fluid behavior, and each choice comes with both advantages and disadvantages.

Advanced simulation methods like FEM fluid simulation or smoothed-particle hydrodynamics approximate the behavior of the fluid quite well, but their computation is quite expensive in both time and memory. Due to their performance characteristics and the current state of computational power, it is difficult to use these methods in the actual drainability analysis. They can, however, be used for validation of a potential solution.

Consequently, previous work creates simplified fluid models and then performs drainability analysis on the reduced problem.

\section{Straight-line Multiple-Particle Model (Yasui Et. all)}

	\subsection{Core Assumptions}

Yasui et. all chose a simplified particle modeling approach with a few core underlying simplifications:

\begin{itemize}
	\item The fluid body at each ``pool'' would be approximated as a single particle
	\item Water particles only travel in straight lines and do not maintain a kinetic state
	\item The workpiece would rotate infinitesimally slow in one direction for an infinite amount of time
\end{itemize}

	\subsection{Modeling Choice Advantages and Disadvantages}

This modeling approach allowed Yusuke et all to reduce the draining problem down to a particle draining problem. Furthermore, particle simulation within the workpiece was greatly simplified; particles always traveled in straight lines and did not maintain kinetic state. When particles collided with parts of the workpiece, they experienced completely inelastic collisions. Lastly, particles leaving an edge or surface were assumed to have no initial velocity.

Thus, particle simulation in this choice of model was reduced to essentially line projection, propagation, and intersection within the workpiece. Because these tests are computationally inexpensive and have a sub-linear runtime, Yusuke et all were able to perform many particle simulations. The end result is that for every possible rotation axis, a particle is simulated out of every concave vertex in the workpiece.

This thoroughness of simulation produces a great set of results that describes the drainability of any given rotation axis out of the Gaussian sphere. Unfortunately, the assumptions needed to attain this fast particle simulation reduce the usability of the results.

Yusuke et all's model assumes infinitesimally slow rotation for an infinite time; while most workpieces will drain fairly quickly, there is no upper bound (or information at all) about the amount of time required for draining. Manufacturers may want to choose an axis that drains the workpiece as quickly as possible, which is not possible under this model.
Furthermore, since workpieces are only rotated in a single direction about the rotation axis, this unnecessarily reduces the number of workpieces that can be drained. A model that allows bi-directional rotation could provide many more rotation axes that fully drain the workpiece.

%	\subsection{Infinitesimally Slow Rotations}

%The assumption about workpiece rotation meant that when particles rolled out of a concave vertex, the gravity direction was always nearly perpendicular to the leading edge.

%\myfigure{prevworkgravityperp}{The gravity direction is perpendicular to concave vertex leading edges.}

%	\subsection{Inelastic Collisions}

%Since water particles do not maintain kinetic state in Yasui et all, collisions against a plane are essentially just projections onto that plane.

%\myfigure{prevworkinelasticcollision}{Demonstration of purely inelastic collisions, where velocities are projected onto the collision surface.}

This work aims to produce results that are directly useful in manufacturing when physically controlling the workpiece rotaton fixture. A direct control sequence as actually produced out of this work that will drain a single particle out of a provided 2D polygonal mesh.

\section{Kinetic Single-Particle Model}

Our particle modeling approach has a few similar but different underlying simplifications:

\begin{itemize}
	\item The fluid in the entire workpiece will be approximated by a single particle
	\item Water particles travel according to a reasonable kinetic model (acceleration, but no aerodynamic drag)
	\item The workpiece can be rotated in either direction along it's rotation axis
	\item The workpiece is a two dimensional polygonal mesh
\end{itemize}

	\subsection{Assumption Effects}

These assumptions lead to work that is complementary to Yusuke et all's work. Their set of assumptions allow for highly performant drainability analysis about a particular axis. This analysis can performed over the entire Gaussian sphere of axes, leading to results that allow designers and manufacturers to pick a rotation axis for a fixture (or detect when a geometry change prevents draining about a given axis).




