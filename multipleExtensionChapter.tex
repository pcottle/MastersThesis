\chapter{Extension to Multiple Particles}

We now present an extension of this work for draining multiple particles out of a workpiece. The initial adaption to our approach is fairly trivial, but the two heuristics we present shortly afterwards dramatically improve search performance.

\section{Initial Adaption}

Only two elements of our approach need to be modified in order to drain multiple particles out of a workpiece. The first is a modification to the state definition; the state is now a set of settled locations, one for each particle:

\myequation{
  (V_{cc}^{1}, V_{cc}^{2}, V_{cc}^{3}, ... , V_{cc}^{n})
} {
  \label{eq:multipleParticleState}
}

The second modification is to our successor function definition; the successor function now takes in a set of particle locations and produces a set of ``reachable'' states:

\myequation{
  successor((V_{cc}^{1}, V_{cc}^{2}, ... , V_{cc}^{n})^i) = \left \{ (V_{cc}^{1}, V_{cc}^{2}, ... , V_{cc}^{n})^1, (V_{cc}^{1}, V_{cc}^{2}, ... , V_{cc}^{n})^2, ...  \right \}
} {
  \label{eq:successorMultipleParticle}
}

With only these modifications alone we are now able to search for solutions that drain multiple particles out of the workpiece.

\subsection{Implementation}

These two modifications to our approach result in a fair number of changes in the implementation. One primary implementation difference is that now a given turn must be simulated across all particle locations. This simulation is usually performed by the transition function:

$$
trans(V_{cc}^{i}, turn_{l}) = V_{cc}^j \qquad or \qquad exit
$$

Earlier we had restricted our action space to only turns that produced motion of the particle and kept the particle on the leading edge of the concave vertex. This means there was only one possible outcome for a turn simulation -- motion of the particle was induced, and that motion began immediately when the turn was performed.

Now that any arbitrary turn can be simulated with a given particle, this guarantee no longer holds. Consequently, we now have four possible simulation scenarios:


\begin{itemize}
\item The particle is offscreen; a turn does not change this particle's state.
\item The turn results in no motion; any linear combination between $g_{start}$ and $g_{end}$ does not produce motion of the particle.
\item Motion of the particle starts immediately when the turn begins; $g_{start}$ is a vector that produces motion.
\item Motion of the particle starts midway through the turn; $g_{start}$ does not produce motion and $g_{end}$ does.
\end{itemize}

Because of these multiple simulation scenarios, the multiple-particle version of the transition function must be much more robust against arbitrary input. While not of great detail for the overall picture of the approach, this presents a rather large obstacle during implementation.

\section{Multiple Particle Time Complexity}

Now that our state is defined as a set of concave vertex locations, the state space is unfortunately exponential in the number of particles, with a base defined as the number of concave vertices in the workpiece:

$$
(num(V_{cc}) + 1)^n
$$

This leads to a time-complexity that is now exponential in the number of concave vertices, rather than linear as in \eqref{eq:bigo}:

\myequation{
O(S \cdot E^2 \cdot V_{cc}^{n})
} {
  \label{eq:bigoTotal}
}

Although we earlier had avoided exponential state spaces and runtimes due to our search formulation, the introduction of multiple particles has yet again vastly expanded our state space.

\section{Search Heuristics}

The one advantage of multiple-particle draining is that it allows for more creativity and flexibility when composing search optimizations.

\subsection{Plan Ranking Heuristic}

The first optimization (and true search heuristic) we implemented was simply a ranking of the partial plans in the priority queue during uniform cost search. Rather than ranking only by cost, we also implemented the following two heuristics:

\begin{itemize}
\item Always prefer plans with a higher number of exited particles, regardless of cost.
\item When the number of exited particles is the same, determine the number of particles in the same location and use this feature to rank along with cost.
\end{itemize}

Utilizing these two heuristics also serves to reduce down the potential state space TODO
%TODO


When ranking plans that have

$$
cost - 10 \cdot numSame
$$

where $numSame$ is the number of particles in the same location produced the best mix between optimality of plans and speed of search.

\subsection{Action Space Heuristic}

The second search optimization we present is centered around sampling from the action space. Recall that for a single particle, we defined our action space to sample from as a fixed set of $g_{start}$ vectors and a range of $g_{end}$ vectors that produce motion of the particle, as seen in Figure \ref{restriction3}.

Now with multiple particles, we have multiple sets of these defined vectors:

\myfigure{width=0.8\linewidth}{multiMinPocket}{Action Space sampling for each concave vertex.}

When sampling turns for our successor function, we may sample from each of these defined spaces. Doing so, however, presents three primary concerns.

\begin{itemize}
\item This action space is fairly large and grows linearly with the number of particles in the workpiece, further increasing our algorithm's time complexity.
\item Although the set of $g_{end}$ vectors for a particular concave vertex have been reduced to keep that particular particle on the leading edge of its concave vertex, this $g_{end}$ vector may cause other particles to leave the surface of the workpiece during rotation. This results in scenario \#4 which causes the sample to be rejected.
\item If a particular turn's $g_{start}$ is
\end{itemize}

Of particular interest are the

